\documentclass{scrartcl}
\usepackage{mm_ws15}
\usepackage{mathabx}
\usetikzlibrary{quotes,angles,arrows,decorations.markings}


\newcommand{\xx}{\vec x}
\newcommand{\yy}{\vec y}
\newcommand{\zz}{\vec z}

\newcommand{\uu}{\mat u}
\newcommand{\vv}{\mat v}
\newcommand{\ww}{\mat w}


\newcommand{\sheetTitle}{Blatt 3, Abgabe 10.11.2015 12:00} 
\begin{document}
\maketitle

\section{Direkte Beweise \points{10}}
Beweisen Sie die folgenden Aussagen. 
Übersetzen Sie außerdem Aussagen, die ausformuliert sind, in formale mathematische Aussagen und umgekehrt.
\begin{subex}
  \item\points{2} Für alle reellen Zahlen $x$, die größer als 1 sind, ist $6x + 3$ größer als $3x + 6$.
  \item\points{2} Das Quadrat aller geraden natürlichen Zahlen ist wieder gerade.
  \item\points{2} Alice ist 16 Jahre alt. Damit ist Alice genau doppelt so alt, wie Bob war, als Alice so alt war, wie es Bob jetzt ist. Bob ist 12 Jahre alt.
  \item\points{2} $\forall k, m \in \NN\colon k + m \le k \times m \implies k \ge 2 \land m \ge 2$
  \item\points{2} $\forall p, q \in \RR\colon (\forall x \in \RR\colon x^2 + px + q > 0) \iff \left( \frac{p^2}{4} < q \right)$ 
\end{subex}

\begin{remark}{Hinweise}
  Für b) benutzen Sie die folgende Definition: $n \in \NN$ heißt gerade, wenn es ein $n' \in \NN$ gibt, sodass $n = 2n'$. Hier noch eine grobe Erläuterung der (möglicherweise) unbekannten Symbole (hierbei sind $A$ und $B$ logische Aussagen)
  \begin{itemize}
    \item $A \land B$ : logisches \textsc{und}; ist genau dann wahr, wenn $A$ und $B$ beide wahr sind
    \item $A \implies B$: Implikation; aus der Wahrheit von Aussage $A$ folgt die Wahrheit von Aussage $B$
    \item $A \iff B$: Äquivalenz; aus der Wahrheit von $A$ folgt die Wahrheit von $B$ und aus der Wahrheit von $B$ folgt die Wahrheit von $A$. 
    Beweist man häufig, indem man die beiden Aussagen $A \implies B$ und $B \implies A$ zeigt.
  \end{itemize}
  Formal exakt werden diese Begriffe im mathematischen Teilgebiet der Aussagenlogik behandelt.
\end{remark}

\begin{solution}[Beispiel]
  Beweisen Sie, dass die Summe zweier gerader Zahlen wieder gerade ist.\\

  \emph{Formal:} $\forall m, n \in \NN; m, n \mbox{ gerade} \implies (m + n) \mbox{ gerade}.$

  \emph{Beweis:} Da $m$ und $n$ gerade sind existieren lt.\ Definition $m', n' \in \NN$, sodass $m = 2m'$ und $n = 2n'$. Damit folgt, dass $m + n$ gerade ist, da
  \[
    m + n = 2m' + 2n' = 2 (m' + n').
  \] 

  Den selben Beweis kann man mit weniger Prosa wie folgt aufschreiben:
  \begin{align*}
    m, n \mbox{ gerade} 
    &\iff \exists m', n' \in \NN\colon m = 2m' \land n = 2n' \\
    &\implies m + n = 2 (m' + n') \\
    &\implies (m + n) \mbox{ gerade}
  \end{align*}
\end{solution}


\section{Euklidisches Skalarprodukt \points{13}}
In der Vorlesung wurde das euklidische Skalarprodukt in $\RR^n$ definiert durch
\[
  \sp{\cdot}{\cdot} \colon \RR^n \times \RR^n \to \RR, (\uu, \vv) \mapsto \sp{\uu}{\vv} = \sum_{k=1}^n u_k v_k.
\]
\begin{subex}
  \item\points{4} Zeigen Sie, dass $\sp{\cdot}{\cdot}$ ein Skalarprodukt auf $\RR^n$ ist, d.h.\ die drei Eigenschaften eines Skalarprodukts erfüllt sind (siehe Vorlesung 1.5.1).
  \item\points{3} Berechnen Sie $\sp{\uu}{\vv}$, $\norm{\uu}$ und $\norm{\vv}$ für 
  \[
    \uu = \colvec{1 \\ 0 \\ 3} \quad\mbox{und}\quad \colvec{-5 \\ 2 \\ 1}.
  \]
  Hier bezeichnet $\norm{\uu} = \sqrt{\sp{\uu}{\uu}}$ die induzierte Norm.
  \item\points{2} Finden Sie alle Vektoren $\vv \in \RR^2$, die mit $\uu = \colvec{1 \\ 2}$ das Skalarprodukt $\sp{\uu}{\vv} = 2$ haben.  
  \item\points{4} Welche der folgenden Beispiele definiert ein Skalarprodukt auf $V$?
  Überprüfen Sie jeweils die Bedingungen an ein Skalarprodukt.
  \begin{itemize}
    \item $V = \RR^3, \sp{\uu}{\vv} = 2u_1 v_1 +u_2 v_2 +5u_3 v_3$
    \item $V = \RR^2, \sp{\uu}{\vv} = u_2 v_2 - u_1 v_1$
    \item $V =\RR^2, \sp{\uu}{\vv} = 2u_1 v_1 + u_1 v_2 + u_2 v_1 + 3u_2 v_2$
    \item $V =\RR^2, \sp{\uu}{\vv} = 2u_1 v_1 + u_1 v_2 + 3u_2 v_1 + u_2 v_2$
  \end{itemize}
\end{subex}


\section{Skalarprodukt\,\&\,Winkel \points{7}}
Der Kosinussatz, welcher Ihnen vielleicht aus der Schule bekannt ist, stellt eine Beziehung in einem Dreieck zwischen den Seiten $a,b,c$ und dem der Seite $c$ gegenüberliegendem Winkel $\varphi$ her.
Er lautet
  \[
  c^2=a^2+b^2-2ab\cos \varphi.
  \]
\begin{subex}
  \item\points{4} Leiten Sie diese Beziehung her, dabei soll Ihnen die Skizze als Inspiration dienen.
  \item\points{2} Benutzen Sie den Kosinussatz um für beliebiges $n \in \NN$ folgendes zu zeigen: Für $\uu, \vv \in \RR^n$ gilt
  \[
    \cos \sphericalangle(\uu,\vv) = \frac{\sp{\uu}{\vv}}{\norm{\uu}\norm{\vv}}.
  \] 
  Hierbei bezeichnet $\sphericalangle(\uu,\vv)$ den von $\uu$ und $\vv$ aufgespannten Winkel.
  \item\points{1} Zeigen Sie, dass zwei Vektoren $\uu, \vv \in \RR^n$, $\uu, \vv \neq 0$ genau dann senkrecht aufeinander stehen, wenn ihr Skalarprodukt verschwindet.
\end{subex}


\vspace{1em}
\begin{center}
  \begin{tikzpicture}[
    scale=1,
    baseline/.style={line width=1pt}
  ]

  \coordinate (A) at (5,0);
  \coordinate (B) at (0,0);
  \coordinate (C) at (3,2);

  \draw (A) -- (B) node[midway, below] {$a$};
  \draw (B) -- (C) node[midway, above left] {$b$};
  \draw[dotted] (C) -- (A) node[midway, above right] {$c$};

  \draw pic["$\varphi$",draw,angle eccentricity=0.6,angle radius=1cm,line width=.5pt] {angle=A--B--C};
  
  \end{tikzpicture}
\end{center}



 

\end{document}
