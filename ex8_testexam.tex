\documentclass{scrartcl}
\usepackage{mm_ws15}
\usepackage{tabularx}
\def\tabularxcolumn#1{m{#1}}

\newcommand{\vv}{\vec{v}}
\newcommand{\uu}{\vec{u}}
\newcommand{\ww}{\vec{w}}

\newcommand{\UU}{\mat{u}}
\newcommand{\VV}{\mat{v}}
\newcommand{\WW}{\mat{w}}


\newcommand{\sheetTitle}{Blatt 8 -- Probeklausur}
\begin{document}
\maketitle

\vspace{\baselineskip}

\centering
\begin{tabularx}{\textwidth}{|X|X|X|l|}
  \hline
  Name & Matrikelnummer & Studienfach & Übungsgruppe \\ 
  \hline
       &                &             &  \\[8ex]
  \hline
\end{tabularx}\\

\vspace{\baselineskip}

\raggedleft
\begin{tabularx}{\textwidth}{|l|*5{>{\centering\arraybackslash}X|}@{}}
  \hline
  Aufgabe & 1 [10] & 2 [10] & 3 [15] & 4 [5] & 5 [10] \\
  \hline
  Punkte  & & & & & \\[3ex]
  \hline
\end{tabularx}

\raggedright
\begin{remark}{Hinweise}
  Sie haben zur Bearbeitung dieser Klausur 90 Minuten Zeit.
  Die Aufgaben sind \emph{nicht} nach Schwierigkeit sortiert, \emph{lesen Sie deshalb zunächst alle Aufgaben gründlich durch} und beginnen Sie mit der für Sie einfachsten Aufgabe.
  Bitte schreiben Sie Ihren Namen leserlich auf dieses Deckblatt und auf alle Zettel, die Sie zur Korrektur abgeben.
  Bitte schreiben Sie nicht mit Bleistift.
  Taschenrechner und ähnliche Hilfsmittel sind nicht erlaubt und auch nicht notwendig. 
  Bitte fangen Sie \emph{jede Aufgabe auf einem neuen Blatt} an. Unterschreiben Sie die Klausur einmal auf dem Deckblatt und einmal am Ende der Klausur.

  \begin{center}
    \large\textbf{Viel Erfolg!} 
  \end{center}
\end{remark}




\raggedright
%%%%%%%%%%%%%%%%%%%%%%%%%%%%%%%%%%%%%%%%%%%%%%%%%%%%%%%%%%%%%%%%%%%%%%%%%%%%%%%%
\section{Pop-Quiz \points{10}}
\label{sec:popquiz}

\begin{subex}
  \item\points{2} Wann sind Vektoren $\vv_1,\ldots,\vv_N \in \RR^d$ linear unabhänging? Wann bilden sie eine Basis?
  \item\points{2} Verschiebungen in der Ebene und der $\RR^n$ ($n \in \NN$) sind Beispiele für welche mathematische Struktur? Nennen Sie ein weiteres.
  \item\points{2} Nenne Sie eine Definition der Ableitung einer Funktion $f\colon \RR \to \RR$ im Punkt $x_0 \in \RR$.
  \item\points{2} Wie können Sie die Koeffizienten $a_k$ der Taylorreihe
  \[
    f(x) = \sum_{k=0}^\infty a_k (x - x_0)^k
  \]
  der Funktion $f\colon \RR \to \RR$ um den Punkt $x_0 \in \RR$ berechnen?
  \item\points{2} Wie lautet die Polardarstellung der komplexen Zahl $z = a + \ii b$ für $a, b \in \RR$ mit $a > 0$ und $b > 0$?
\end{subex}


%%%%%%%%%%%%%%%%%%%%%%%%%%%%%%%%%%%%%%%%%%%%%%%%%%%%%%%%%%%%%%%%%%%%%%%%%%%%%%%%
\section{Gram-Schmidt Orthogonalisierung \points{10}}
\label{sec:gram_schmidt_orthogonalisierung}

Betrachte die folgenden Vektoren in $\RR^3$ gegeben durch ihre Komponenten bzgl. der Standardbasis:
\[
  \UU = \colvec{2 \\ 0 \\ 2}, \quad \VV = \colvec{1 \\ 1 \\ 1}, \quad{\WW} = \colvec{3 \\ 2 \\ 1}
\]

\begin{subex}
  \item\points{2} Sind $\uu$, $\vv$, $\ww$ orthonormal? Sind sie orthogonal?
  \item\points{5} Berechnen Sie die Komponenten der folgenden Vektoren bezüglich der Standardbasis
  \begin{align*}
    &\tilde\uu = \frac{\uu}{\Vert \uu \Vert} & \\
    &\vv' = \vv - \langle \tilde \uu, \vv \rangle \tilde\uu 
    &\tilde\vv = \frac{\vv'}{\Vert \vv' \Vert} \\
    &\ww' = \ww - \langle \tilde \uu, \ww \rangle \tilde\uu - \langle \tilde \vv, \ww \rangle \tilde\vv 
    &\tilde\ww = \frac{\ww'}{\Vert \ww' \Vert}
  \end{align*}
  \item\points{3} Zeigen Sie, dass $\tilde\uu$, $\tilde\vv$, $\tilde\ww$ orthonormal sind.
\end{subex}



%%%%%%%%%%%%%%%%%%%%%%%%%%%%%%%%%%%%%%%%%%%%%%%%%%%%%%%%%%%%%%%%%%%%%%%%%%%%%%%%
\section{Differentiation \& Integration \points{15}}
\label{sec:differentiation}

Sei $f(x) = \cos x \log x^2$ und $g(x, y) = y \frac{1 - x}{1 + x}$ und $h(x) = \sin x \cos x$

\begin{subex}
  \item\points{4} Berechnen Sie $f'(x)$ und $f''(x)$.
  \item\points{6} Berechnen Sie $\pderiv[g]{x}(x, y)$, $\pderiv[g]{y}(x, y)$ und $\frac{\partial^2 g}{\partial x \partial y}(x, y)$.
  \item\points{5} Berechnen Sie \emph{die Fläche}, die $h$ im Intervall $[-\frac{\pi}{2}, \frac{\pi}{2}]$ mit der $x$-Achse einschließt.
\end{subex} 


%%%%%%%%%%%%%%%%%%%%%%%%%%%%%%%%%%%%%%%%%%%%%%%%%%%%%%%%%%%%%%%%%%%%%%%%%%%%%%%%
\section{Gaussintegral \points{5}}
\label{sec:gaussintegral}

Berechnen Sie das 1-dimensionale Gaussintegral
\[
  G = \int_{-\infty}^{\infty} \dd x \,  \ee^{-\frac{x^2}{2}} 
\]
\begin{remark}{Hinweis}
  Schreiben Sie $G^2$ als 2D Integral über $(x, y)$ und führen Sie eine Transformation auf Polarkoordinaten $(r, \phi)$ durch. 
\end{remark} 
% Gaussintegral


%%%%%%%%%%%%%%%%%%%%%%%%%%%%%%%%%%%%%%%%%%%%%%%%%%%%%%%%%%%%%%%%%%%%%%%%%%%%%%%%
\section{Komplexe Zahlen \points{10}}
\label{sec:komplexe_zahlen}

\begin{subex}
  \item\points{5} Sei $z = \ee^{-\ii \frac{\pi}{4}}$ und $w = 2 + \ii$, berechnen Sie $z + w$, $zw$, $z(\bar{z} - w^2)$, $\frac{z}{w}$ und $z^3$. Geben Sie das Ergebnis in der Form $a + \ii b$ an, wobei $a, b \in \RR$.
  \item\points{2} Skizzieren Sie $z_n = \ee^{-\ii \frac{\pi n}{4}}$ für $n \in \NN$ in der Gaussschen Zahlenebene. 
  \item\points{3} Benutzen Sie Eulers Formel um die folgenden trigonometrischen Formeln zu beweisen
  \[
    \cos^2 x + \sin^2 x = 1 \quad\mbox{und}\quad \tan \frac{x}{2} = \frac{\sin x}{1 + \cos x}
  \]
\end{subex}


\end{document}
