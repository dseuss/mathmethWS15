\documentclass{scrartcl}
\usepackage{mm_ws15}
\usepackage{wrapfig}

\newcommand{\permN}{\mathcal{S}_n}
\renewcommand{\mod}{\operatorname{mod}}

% TODO Insert turn-in date
\newcommand{\sheetTitle}{Blatt 1, Abgabe ???}
\newcommand{\uu}{\vec{u}}
\newcommand{\vv}{\vec{v}}
\newcommand{\ww}{\vec{w}}
\newcommand{\ee}{\vec{e}}

\begin{document}
\maketitle

\section{Gruppe der Permutationen}
\label{groupperm}
Als Beispiel für eine Gruppe wurde in der Vorlesung die Gruppe der Permutationen von $n$ Elementen $(\permN{}, \ast)$ genannt.
Hierbei ist $\permN$ die Menge aller Anordnungen der Menge $\NN_n = \{1,\ldots,n\}$ (der Menge der natürlichen Zahlen von $1$ bis $n$).
Genauer: $\permN$ ist die Menge der bijektiven (eineindeutigen) Abbildungen von $\NN_n$ auf sich selber.
Die Operation $\ast$ ist definiert durch Hintereinanderausführung der Abbildungen aus $\permN$.
Um die Notation zu vereinfachen werden wir im Weiteren eine Permutation $\sigma \in \permN$ auch als Tupel schreiben $\sigma = (\sigma(1), \sigma(2), \ldots, \sigma(n))$.
Diese Schreibweise spiegelt die äquivalente Charakterisierung von $\permN$ als Menge aller Anordnungen von $n$ Elementen wider.
\begin{subex} 
  \item Für $n=4$ betrachte $\sigma_1 = (2, 4, 1, 3)$, $\sigma_2 = (3, 2, 4 , 1)$ und $\sigma_3 = (2, 1, 3, 2)$.
  Welcher dieser Abbildungen ist keine Permutation?
  Warum?
  Was ist $\sigma_1 \ast \sigma_2$ und $\sigma_2 \ast \sigma_1$?
  Was schlussfolgern Sie daraus für die Gruppe $(\permN,\ast)$?
  \item Beweisen Sie, dass für beliebiges $n \in \NN$, dass $(\permN,\ast)$ eine Gruppe ist.
  \item Beschreiben Sie einen Algorithmus (eine explizite Konstruktionsvorschrift) für $\permN$.
  Mit anderen Worten: gegeben eine Zahl $n$, wie würden Sie alle Elemente aus $\permN$ finden?
  Benutzen Sie \emph{Ihre eigene} Vorschrift um $\mathcal{S}_3$ zu konstruieren!
\end{subex}
\begin{remark}{Hinweis}
  In \ref{groupperm}b) ist die Definition als bijektive Abbildungen von $\NN_n$ nach $\NN_n$ vorteilhaft, für  \ref{groupperm}c) die äquivalente Charakterisierung als Menge aller Anordnungen von $n$ Objekten.
  Beachten Sie für c), dass $\permN$ genau $n! = n \times (n-1) \times \cdots \times 2 \times 1$ Elemente hat.
  Wie kommen die Faktoren zustande?
\end{remark}


\section{Modulare Arithmetik}

Die Menge $\NN_n = \{1,\ldots,n\}$ versehen mit der üblichen Addition $+$ von natürlichen Zahlen bildet keine Gruppe.
Stattdessen definieren wir die \quotes{Addition modulo $n$} für $a, b \in \NN_n$
\[
  a \oplus b = \mod_n(a + b).
\]
Für festes $n \in \NN$ ordnet die Modulo-Operation jeder natürlichen Zahl $a \in \NN$ den ganzzahligen Rest der Division $a / n$ zu.
Mit anderen Worten: für jedes $a \in \NN$ existiert genau ein $k \in \{0, 1, \ldots \}$ und $\mod_n(a) \in \NN_n$, sodass
\[
  a = k \times n + \mod_n(a).
\]

Diese Konstruktion kann man sich für $n=7$ an untenstehender Abbildung vor Augen führen.
Die außerhalb des Kreises aufgetragenen Zahlen erhält man durch \quotes{Aufwickeln} des Zahlenstrahls der natürlichen Zahlen:
Im Ursprung des Zahlenstrahls bei 1 startend geht man $n$ Schritte auf dem Zahlenstrahl vorwärts und identifiziert die Punkte ($1$, $n+1$), ($2$, $n+2$) usw.
Dieses Konstruktion wiederholt man immer wieder nach $n$ Schritten -- man \quotes{wickelt} also Segmente der Länge $n$ des Zahlenstrahls übereinander.

Das Ergebnis von $\mod_n(a)$ ist dann die Zahl in $\{1,\ldots,n\}$, die man mit $a$ identifiziert hat.
Im Bild sind diese im Inneren des Kreises aufgetragen.
Hier noch einige Beispiele:
\begin{align*}
  \mod_3(2) = \mod_3(5)=2 = \mod_3(8)= 2 \\
  \mod_4(2) = 2, \mod_4(5) = 1, \mod_4(8) = 4\\
  \mod_{10}(17) = 7, \mod_{10}(1321) = 1.
\end{align*}
  
\begin{subex}
  \item Erklären Sie, warum $(\NN_n, +)$ keine Gruppe bildet.
  Ist $(\{0, \ldots, n\},+)$ eine Gruppe?
  \item Sei $n=10$, berechnen Sie $1 \oplus 2$, $3 \oplus 10$, $4 \oplus 7$ und $123 \oplus 456$.
  \item Beweisen Sie, dass $(\NN_n,\oplus)$ eine Gruppe ist.
  Was ist das neutrale Element?
  Wie lautet das inverse Element für $a\in\NN_n$?
  \item Mit welcher in der Vorlesung bereits behandelten Gruppe können Sie $(\NN_n,\oplus)$ identifizieren?
\end{subex}

\begin{center}
  \begin{tikzpicture}[scale=1.5]
  \pgfmathsetmacro{\N}{7};
  \pgfmathsetmacro{\offset}{1.7};
  \pgfmathsetmacro{\inset}{.7};


  \foreach \i in {1,...,\N}{
    \pgfmathsetmacro{\myphi}{360 / \N * (\i - 1)};
    \pgfmathsetmacro{\j}{\i + \N}

    \node at (\offset * cos \myphi,\offset * sin \myphi) {${\i}, {\pgfmathparse{\i + \N}\pgfmathprintnumber{\pgfmathresult}}, \ldots$};
    \node at (\inset * cos \myphi,\inset * sin \myphi) {${\i}$};
    \draw[-|] (cos \myphi,sin \myphi) arc [start angle=\myphi,end angle=(\myphi + 360 / \N),radius=1cm];
  }  
  \end{tikzpicture}
\end{center}

\section{Verschiebungen in der Ebene}
Die in der unten stehenden Abbildung gezeigten Pfeile repräsentieren Vektoren, die hier als Verschiebungen in der Ebene zu interpretieren sind.
\begin{subex}
  \item Skizzieren Sie folgende Linearkombinationen: $\uu + \vv$, $\uu - \ww$, $\frac{1}{2}\uu$, $2\ww + 3(\uu + \vv + \frac{1}{3}\ww)$
  \item Bestimmen Sie die Komponenten $\mat{u}$, $\mat{v}$ und $\mat{w}$ von $\uu$, $\vv$ und $\ww$ bezüglich des Basissystems $(\ee_1, \ee_2)$.
  \item Bestimmen Sie die Komponenten $\mat{e}_2$ von $\ee_2$ bezüglich des Basissystems $(\ee_1, \ww)$.
  \item Warum ist $(\ee_1, \uu)$ kein geeignetes Basissystem für alle Verschiebungen in der Ebene? 
\end{subex}
\begin{center}
  \begin{tikzpicture}[%
    coordline/.style={dotted},
    vector/.style={->}
  ]
      
  \drawgrid{-1}{4}{-1}{2}
  \draw[vector] (1, 0) -- node[above] {$\uu$} ++ (2,0);
  \draw[vector] (3, 2) -- node[above] {$\vv$} ++ (-2,-1);
  \draw[vector] (0, 1) -- node[above left] {$\ww$} ++ (1,1);

  \draw[vector] (-1, -1) -- node[below] {$\ee_1$} ++ (1,0);
  \draw[vector] (-1, -1) -- node[left] {$\ee_2$} ++ (0,1);

  \end{tikzpicture}
\end{center}

\end{document}