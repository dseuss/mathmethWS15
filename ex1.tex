\documentclass{scrartcl}
\usepackage{mm_ws15}
\usepackage{wrapfig}
\usepackage{mathrsfs}


\newcommand{\permN}{\mathcal{S}_n}
\renewcommand{\mod}{\operatorname{mod}}

\newcommand{\sheetTitle}{Blatt 1, Abgabe 26.10.2015 12:00}
\newcommand{\uu}{\vec{u}}
\newcommand{\vv}{\vec{v}}
\newcommand{\ww}{\vec{w}}
\newcommand{\ee}{\vec{e}}

\definecolor{uuuuuu}{rgb}{0.26666666666666666,0.26666666666666666,0.26666666666666666}

\begin{document}
\maketitle

\section{Gruppe der Permutationen \points{10}}
\label{groupperm}
Als Beispiel für eine Gruppe wurde in der Vorlesung die Gruppe der Permutationen von $n$ Elementen $(\permN{}, \ast)$ genannt.
Hierbei ist $\permN$ die Menge aller Anordnungen der Menge $\NN_n = \{1,\ldots,n\}$ (der Menge der natürlichen Zahlen von $1$ bis $n$).
Genauer: $\permN$ ist die Menge der bijektiven (eineindeutigen) Abbildungen von $\NN_n$ auf sich selber.
Die Operation $\ast$ ist definiert durch Hintereinanderausführung der Abbildungen aus $\permN$.
Um die Notation zu vereinfachen werden wir im Weiteren eine Permutation $\sigma \in \permN$ auch als eine geordnete Liste (sogenanntes Tupel) schreiben $\sigma = (\sigma(1), \sigma(2), \ldots, \sigma(n))$.
Diese Schreibweise spiegelt die äquivalente Charakterisierung von $\permN$ als Menge aller Anordnungen von $n$ Elementen wider.
\begin{subex}
  \item\points{3} Für $n=4$ betrachte $\sigma_1 = (2, 4, 1, 3)$, $\sigma_2 = (3, 2, 4 , 1)$ und $\sigma_3 = (2, 1, 3, 2)$.
  Welcher dieser Abbildungen ist keine Permutation?
  Warum?
  Was ist $\sigma_1 \ast \sigma_2$ und $\sigma_2 \ast \sigma_1$?
  Was schlussfolgern Sie daraus für die Gruppe $(\permN,\ast)$?
  \item\points{5} Beweisen Sie, dass für beliebiges $n \in \NN$, dass $(\permN,\ast)$ eine Gruppe ist.
  \item\points{2} Beschreiben Sie einen Algorithmus (eine explizite Konstruktionsvorschrift) für $\permN$.
  Mit anderen Worten: gegeben eine Zahl $n$, wie würden Sie alle Elemente aus $\permN$ finden?
  Benutzen Sie \emph{Ihre eigene} Vorschrift um $\mathcal{S}_3$ zu konstruieren!
\end{subex}
\begin{remark}{Hinweis}
  In \ref{groupperm}b) ist die Definition als bijektive Abbildungen von $\NN_n$ nach $\NN_n$ vorteilhaft, für  \ref{groupperm}c) die äquivalente Charakterisierung als Menge aller Anordnungen von $n$ Objekten.
  Beachten Sie für c), dass $\permN$ genau $n! = n \times (n-1) \times \cdots \times 2 \times 1$ Elemente hat.
  Wie kommen die Faktoren zustande?
\end{remark}


\section{Modulare Arithmetik \points{10}}

Die Menge $\NN_n = \{1,\ldots,n\}$ versehen mit der üblichen Addition $+$ von natürlichen Zahlen bildet keine Gruppe.
Stattdessen definieren wir die \quotes{Addition modulo $n$} für $a, b \in \NN_n$
\[
  a \oplus b = \mod_n(a + b).
\]
Für festes $n \in \NN$ ordnet die Modulo-Operation jeder natürlichen Zahl $a \in \NN$ den ganzzahligen Rest der Division $a / n$ zu.
Mit anderen Worten: für jedes $a \in \NN$ existiert genau ein $k \in \{0, 1, \ldots \}$ und $\mod_n(a) \in \NN_n$, sodass
\[
  a = k \times n + \mod_n(a).
\]

Diese Konstruktion kann man sich für $n=5$ an untenstehender Abbildung vor Augen führen.
Man stelle sich den Zahlenstrahl -- im Bild durch die dicke Linie dargestellt -- so aufgewickelt vor, dass immer $i$, $i + n$, $i + 2n$,\ldots{} auf einem der $n$ gestrichelten Strahlen liegen.
Das Ergebnis von $\mod_n(i)$ erhält man, indem man dem Strahl durch $i$ bis zu dem Punkt nach innen folgt, der am nächsten zum Ursprung liegt, d.h.\ alle Zahlen, die auf dem gleichen Strahl liegen haben den gleichen Modulo.
Dieser ist im Bild zur Veranschaulichung eingekreist.
Hier noch einige Beispiele:
\begin{align*}
  \mod_3(2) = \mod_3(5)=2 = \mod_3(8)= 2 \\
  \mod_4(2) = 2, \mod_4(5) = 1, \mod_4(8) = 4\\
  \mod_{10}(17) = 7, \mod_{10}(1321) = 1.
\end{align*}

\newpage
\begin{subex}
  \item\points{1} Erklären Sie, warum $(\NN_n, +)$ keine Gruppe bildet.
  Ist $(\{0, \ldots, n\},+)$ eine Gruppe?
  \item\points{2} Sei $n=10$, berechnen Sie $1 \oplus 2$, $3 \oplus 10$, $4 \oplus 7$ und $123 \oplus 456$.
  \item\points{6} Beweisen Sie, dass $(\NN_n,\oplus)$ eine Gruppe ist.
  Was ist das neutrale Element?
  Wie lautet das inverse Element für $a\in\NN_n$?
  \item\points{1} Mit welcher in der Vorlesung bereits behandelten Gruppe können Sie $(\NN_n,\oplus)$ identifizieren?
\end{subex}

\begin{center}
  \begin{tikzpicture}[scale=0.7,
    number/.style={fill=white},
    basenumber/.style={circle,draw,fill=white}
  ]
    %\clip(-7,-7) rectangle (8,8);

    \draw[->,line width=1.2pt, smooth,samples=100,domain=0.0:14.0] plot[parametric] function{(t/4.0+2.0)*cos((t)),(t/4.0+2.0)*sin((t))};

    \draw [->,dash pattern=on 2pt off 2pt,domain=0.0:7] plot(\x,{(-0.-0.*\x)/1.});
    \draw [->,dash pattern=on 2pt off 2pt,domain=0.0:2] plot(\x,{(-0.--0.9510565162951531*\x)/0.30901699437494734});
    \draw [<-,dash pattern=on 2pt off 2pt,domain=-6:0.0] plot(\x,{(-0.--0.5877852522924732*\x)/-0.8090169943749473});
    \draw [<-,dash pattern=on 2pt off 2pt,domain=-6:0.0] plot(\x,{(-0.-0.5877852522924728*\x)/-0.8090169943749472});
    \draw [->,dash pattern=on 2pt off 2pt,domain=0.0:2] plot(\x,{(-0.-0.9510565162951536*\x)/0.30901699437494723});


    \node [basenumber] at (2.,0.) {$1$};
    \node [number] at (3.5707962750845654,-7.38587137808993E-7) {$6$};
    \node [number] at (5.141592699769043,9.497413887440426E-7) {$11$};
    \node [basenumber] at (0.7151145872951262,2.2008962281510116) {$2$};
    \node [number] at (1.2005181769307902,3.694811982928024) {$7$};
    \node [number] at (1.6859191319071674,5.188728763108593) {$12$};
    \node [basenumber] at (-2.126354241530161,1.5448870026154518) {$3$};
    \node [number] at (-3.3971558936085913,2.468177040389493) {$8$};
    \node [basenumber] at (-2.3805147208547806,-1.7295447587396982) {$4$};
    \node [number] at (-3.65131495578304,-2.6528367592821107) {$9$};
    \node [basenumber] at (1.006355674150686,-3.09724602495394) {$5$};
    \node [number] at (1.4917598588409122,-4.591161738390655) {$10$};
  \end{tikzpicture}
\end{center}

\section{Verschiebungen in der Ebene \points{8}}
Die in der unten stehenden Abbildung gezeigten Pfeile repräsentieren Vektoren, die hier als Verschiebungen in der Ebene zu interpretieren sind.
\begin{subex}
  \item\points{2} Skizzieren Sie folgende Linearkombinationen: $\uu + \vv$, $\uu - \ww$, $\frac{1}{2}\uu$, $2\ww + 3(\uu + \vv + \frac{1}{3}\ww)$
  \item\points{3} Bestimmen Sie die Komponenten $\mat{u}$, $\mat{v}$ und $\mat{w}$ von $\uu$, $\vv$ und $\ww$ bezüglich des Basissystems $(\ee_1, \ee_2)$.
  \item\points{2} Bestimmen Sie die Komponenten $\mat{e}_2$ von $\ee_2$ bezüglich des Basissystems $(\ee_1, \ww)$.
  \item\points{1} Warum ist $(\ee_1, \uu)$ kein geeignetes Basissystem für alle Verschiebungen in der Ebene?
\end{subex}
\begin{center}
  \begin{tikzpicture}[%
    coordline/.style={dotted},
    vector/.style={->}
  ]

  \drawgrid{-1}{4}{-1}{2}
  \draw[vector] (1, 0) -- node[above] {$\uu$} ++ (2,0);
  \draw[vector] (3, 2) -- node[above] {$\vv$} ++ (-2,-1);
  \draw[vector] (0, 1) -- node[above left] {$\ww$} ++ (1,1);

  \draw[vector] (-1, -1) -- node[below] {$\ee_1$} ++ (1,0);
  \draw[vector] (-1, -1) -- node[left] {$\ee_2$} ++ (0,1);

  \end{tikzpicture}
\end{center}

\end{document}
