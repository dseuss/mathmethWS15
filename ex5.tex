\documentclass{scrartcl}
\usepackage{mm_ws15}

\newcommand{\sheetTitle}{Blatt 5, Abgabe 24.11.2015 12:00}

\newcommand\smallO[1]{
        \mathchoice
            {% \displaystyle
                \ensuremath{\mathop{}\mathopen{}{\scriptstyle\mathcal{O}}\mathopen{}\left(#1\right)}
            }
            {% \textstyle
                \ensuremath{\mathop{}\mathopen{}{\scriptstyle\mathcal{O}}\mathopen{}\left(#1\right)}
            }
            {% \scriptstyle
                \ensuremath{\mathop{}\mathopen{}{\scriptscriptstyle\mathcal{O}}\mathopen{}\left(#1\right)}
            }
            {% \scriptscriptstyle  
                \ensuremath{\mathop{}\mathopen{}{o}\mathopen{}\left(#1\right)}
            }
    }
\begin{document}
\maketitle


%%%%%%%%%%%%%%%%%%%%%%%%%%%%%%%%%%%%%%%%%%%%%%%%%%%%%%%%%%%%%%%%%%%%%%%%%%%%%%%%
\section{Ableitungen elementarer Funktionen \points{8}}
\label{sec:ableitungen_elementarer_funktionen}


In der Vorlesung haben Sie die Definition der \emph{Ableitung einer Funktion am Punkt $x$} über den Grenzwert des Differentialquotienten kennengelernt
\[
  f'(x) := \lim_{\delta x \to 0}  \frac{f(x + \delta x) - f(x)}{\delta x}.
\] 
Dabei setzten wir natürlich voraus, das der Grenzwert existiert.
Benutzen Sie die obige Definition, um die folgenden elementaren Ableitungen zu zeigen

\begin{subex}
  \item\points{2} Für die konstante Funktion $f(x) = a$ mit $a \in \RR$ gilt $f'(x) = 0$
  \item\points{3} $\left( x^n \right)' = n \, x^{n-1}$ für $n \in \NN$ (Hinweis: Binomischer Lehrsatz)
  \item\points{3} $\left( \frac{1}{x} \right)' = -\frac{1}{x^2}$
\end{subex}


%%%%%%%%%%%%%%%%%%%%%%%%%%%%%%%%%%%%%%%%%%%%%%%%%%%%%%%%%%%%%%%%%%%%%%%%%%%%%%%%
\section{Weitere Ableitungen \points{10}}
\label{sec:weitere_ableitungen}

Bestimmen Sie die Ableitungen der folgenden Funktionen.
Sie können dafür die bekannten elementaren Ableitungen sowie die üblichen Rechenregeln verwenden (Linearität, Produktregel, Kettenregel, Umkehrregel, etc.).

\begin{subex*}
  \item $a x^2 + b x + c$
  \item $\sin (x^2 + 5)$
  \item $\frac{1 + x}{1 - x}$
  \item $\cos^2 (x^2)$
  \item $\log x$
  \item $x^x$
\end{subex*}

\begin{remark}{Hinweise}
  Der Logarithmus ist definiert als Inverses der Exponentialfunktion $\mathrm{e}^x$.
  Es gilt $(\sin(x))' = \cos(x)$ sowie $\left( \mathrm{e}^x \right)' = \mathrm{e}^x$.
  Die Potenz mit beliebigen Exponenten ist definiert über den Logarithmus:
  \[
    a^x := \mathrm{e}^{x \log a}
  \]
  Für $x \in \mathbb{Z}$ stimmt diese Definition mit der Ihnen bereits bekannten überein. 
\end{remark}

%%%%%%%%%%%%%%%%%%%%%%%%%%%%%%%%%%%%%%%%%%%%%%%%%%%%%%%%%%%%%%%%%%%%%%%%%%%%%%%%
\section{Partielle Ableitungen \points{8}}
\label{sec:partielle_ableitungen}
Bisher haben wir nur Funktionen einer Veränderlichen betrachtet; genauso kann eine Funktion natürlich von zwei oder mehr Variablen abhängen. 
Betrachtet man nun eine kleine Änderung in einer der Variablen und möchte wissen, wie dies den Funktionswert beeinflusst, benötigt man das Konzept der \emph{partiellen Ableitung}, welches Sie in der Vorlesung kennengelernt haben.\\
Betrachten Sie die folgenden Funktionen:
\begin{align*}
f(x_1,x_2)=\sin(x_1)+x_2^2 \qquad g(y_1,y_2,y_3)= y_1\cdot\exp(2\cdot y_3)-x_1^2\cdot \ln(y_2).
\end{align*}

\begin{subex}
\item\points{5} Berechnen Sie die partiellen Ableitungen $\frac{\partial}{\partial x_1}f(x_1,x_2)$, $\frac{\partial}{\partial x_2}f(x_1,x_2)$, $\frac{\partial}{\partial y_1} g(y_1,y_2,y_3)$, $\frac{\partial}{\partial y_2} g(y_1,y_2,y_3)$ und $\frac{\partial}{\partial y_3}g(y_1,y_2,y_3)$.
\item\points{3} Nun betrachten wir die zweifache Ableitung von $f$ nach unterschiedlichen Variablen. Berechnen Sie $\frac{\partial^2}{\partial x_1\partial x_2}f(x_1,x_2)$ und $\frac{\partial^2}{\partial x_2\partial x_1}f(x_1,x_2)$.
Spielt die Reihenfolge, in der man die Ableitungen ausführt, eine Rolle?
Erklären Sie!
\end{subex}

%%%%%%%%%%%%%%%%%%%%%%%%%%%%%%%%%%%%%%%%%%%%%%%%%%%%%%%%%%%%%%%%%%%%%%%%%%%%%%%%
\section{Partielle Ableitung als lineare Approximation \points{4}}
\label{sec:partielle_ableitung_als_lineare_approximation}

Leiten Sie Formel~\eqref{eq:twoseventeen} der Vorlesung\footnote{%
  Evtl. fehlte in Ihrer Version des Skripts der Term $f(x_i, x_j)$.
} her:
\begin{equation}
  \tag{2.17}
  \label{eq:twoseventeen}
  f(x_i + \delta x_i, x_j + \delta x_j) = f(x_i, x_j) + \delta x_i \pderiv[f]{x_i}(x_i, x_j) + \delta x_j \pderiv[f]{x_j}(x_i, x_j) + \mathcal{O}(\delta x_{i/j}^2)
\end{equation}
\end{document}
