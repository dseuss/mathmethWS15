\documentclass{scrartcl}
\usepackage{mm_ws15}

\newcommand{\sheetTitle}{Blatt 5, Abgabe 24.11.2015 12:00}
\begin{document}
\maketitle


%%%%%%%%%%%%%%%%%%%%%%%%%%%%%%%%%%%%%%%%%%%%%%%%%%%%%%%%%%%%%%%%%%%%%%%%%%%%%%%%
\section{Ableitungen elementarer Funktionen}
\label{sec:ableitungen_elementarer_funktionen}


In der Vorlesung haben Sie die Definition der \emph{Ableitung einer Funktion am Punkt $x$} über den Grenzwert des Differentialquotienten kennengelernt
\[
  f'(x) := \lim_{\delta x \to 0}  \frac{f(x + \delta x) - f(x)}{\delta x}.
\] 
Dabei setzten wir natürlich voraus, das der Grenzwert existiert.
Benutzen Sie die obige Definition, um die folgenden elementaren Ableitungen zu zeigen

\begin{subex}
  \item Für die konstante Funktion $f(x) = a$ mit $a \in \RR$ gilt $f'(x) = 0$
  \item $\left( x^n \right)' = n \, x^{n-1}$ für $n \in \NN$ (Hinweis: Binomischer Lehrsatz)
  \item $\left( \frac{1}{x} \right)' = -\frac{1}{x^2}$
\end{subex}



%%%%%%%%%%%%%%%%%%%%%%%%%%%%%%%%%%%%%%%%%%%%%%%%%%%%%%%%%%%%%%%%%%%%%%%%%%%%%%%%
\section{Ableitungsregeln}
\label{sec:ableitungsregeln}

In dieser Aufgabe sollen Sie die wichtigsten Regeln zu Berechnung der Ableitung bewiesen werden.
Benutzen Sie dafür die Definition aus Aufgabe~\ref{sec:ableitungen_elementarer_funktionen}.
Dabei sind $f, g$ zwei (in den entsprechenden Punkten) differenzierbare, reelle Funktionen und $a \in \RR$ eine Konstante.

Bereits bewiesene Aussagen (auch aus Aufgabe~\ref{sec:ableitungen_elementarer_funktionen}) können selbstverständlich verwendet werden.
Die Reihenfolge, in der die Aufgaben behandelt werden, ist frei wählbar.

\begin{subex}
  \item $(a f)'(x) = a f'(x)$ und $(f + g)'(x) = f'(x) + g'(x).$ \emph{(Linearität der Ableitung)}
  \item $(f g)'(x) = f'(x) g(x) + f(x) g'(x)$ \emph{(Produktregel)}
  \item $\left( \frac{f}{g} \right)'(x) = \frac{f'(x) g(x) + f(x) g'(x)}{g(x)^2}$ \emph{(Quotientenregel)}
  \item $(g \circ f)'(x) = g'(f(x)) f'(x)$ \emph{(Kettenregel)}\\
  Hierbei ist $\circ$ die Verkettung zweier Funktionen: $(g \circ f)(x) = g(f(x))$
  \item(3 Zusatzpunkte) $(f^{-1})'(x) = \frac{1}{f'\left( f^{-1}(x) \right)}$ \emph{(Ableitung der Umkehrfunktion)}\\
  Hierbei muss $f$ an den entsprechenden Punkten invertierbar sein.
  Beachten Sie die Definition des Inversen: $y = f(x) \iff f^{-1}(y) = x$, was folgt daraus für $f^{-1}(f(x))$?
\end{subex}


%%%%%%%%%%%%%%%%%%%%%%%%%%%%%%%%%%%%%%%%%%%%%%%%%%%%%%%%%%%%%%%%%%%%%%%%%%%%%%%%
\section{Weitere Ableitungen}
\label{sec:weitere_ableitungen}

Bestimmen Sie die Ableitungen der folgenden Funktionen.
Dabei können alle Ergebnisse aus den vorausgegangen Aufgaben sowie die Hinweise verwendet werden.
Bitte geben Sie bei jedem Rechenschritt genau an, was Sie verwenden.

\begin{subex*}
  \item $a x^2 + b x + c$
  \item $\sqrt{x}$
  \item $\frac{1 + x}{1 - x}$
  \item $\sin (x^2 + 5)$
  \item $\log x$
  \item $x^x$
\end{subex*}

\begin{remark}{Hinweise}
  Der Logarithmus ist definiert als Inverses der Exponentialfunktion $\mathrm{e}^x$.
  Es gilt $(\sin(x))' = \cos(x)$ sowie $\left( \mathrm{e}^x \right)' = \mathrm{e}^x$.
  Die Potenz mit beliebigen Exponenten ist definiert über den Logarithmus:
  \[
    a^x := \mathrm{e}^{x \log a}
  \]
  Für $x \in \mathbb{Z}$ stimmt diese Definition mit der Ihnen bereits bekannten überein. 
\end{remark}


%%%%%%%%%%%%%%%%%%%%%%%%%%%%%%%%%%%%%%%%%%%%%%%%%%%%%%%%%%%%%%%%%%%%%%%%%%%%%%%%
\section{Bahnkurve eines Teilchens}
\label{sec:bahnkurve_eines_teilchens}

Ein Massepunkt bewegt sich auf der Bahnkurve
\[
  \mat r(t) = \colvec{v_x t + x_0 \\ y_0 \\ -\frac{g}{2} t^2 + v_z t + z_0}.
\]
Berechnen Sie seine Geschwindigkeit $\mat v(t) = \frac{\dd \mat r(t)}{\dd t}$ und Beschleunigung $\mat a(t) = \frac{\dd \mat v(t)}{\dd t}$.
Interpretieren Sie das Ergebnis.
\end{document}
