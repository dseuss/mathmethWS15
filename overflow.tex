\documentclass{scrartcl}
\usepackage{mm_ws15}

\newcommand{\sheetTitle}{Aufgaben, die nie gestellt wurden}
\begin{document}
\maketitle


%%%%%%%%%%%%%%%%%%%%%%%%%%%%%%%%%%%%%%%%%%%%%%%%%%%%%%%%%%%%%%%%%%%%%%%%%%%%%%%%
\section{Ableitungsregeln}
\label{sec:ableitungsregeln}

In dieser Aufgabe sollen Sie die wichtigsten Regeln zu Berechnung der Ableitung bewiesen werden.
Benutzen Sie dafür die Definition aus Aufgabe~\ref{sec:ableitungen_elementarer_funktionen}.
Dabei sind $f, g$ zwei (in den entsprechenden Punkten) differenzierbare, reelle Funktionen und $a \in \RR$ eine Konstante.

Bereits bewiesene Aussagen (auch aus Aufgabe~\ref{sec:ableitungen_elementarer_funktionen}) können selbstverständlich verwendet werden.
Die Reihenfolge, in der die Aufgaben behandelt werden, ist frei wählbar.

\begin{subex}
  \item $(a f)'(x) = a f'(x)$ und $(f + g)'(x) = f'(x) + g'(x).$ \emph{(Linearität der Ableitung)}
  \item $(f g)'(x) = f'(x) g(x) + f(x) g'(x)$ \emph{(Produktregel)}
  \item $\left( \frac{f}{g} \right)'(x) = \frac{f'(x) g(x) + f(x) g'(x)}{g(x)^2}$ \emph{(Quotientenregel)}
  \item $(g \circ f)'(x) = g'(f(x)) f'(x)$ \emph{(Kettenregel)}\\
  Hierbei ist $\circ$ die Verkettung zweier Funktionen: $(g \circ f)(x) = g(f(x))$
  \item(3 Zusatzpunkte) $(f^{-1})'(x) = \frac{1}{f'\left( f^{-1}(x) \right)}$ \emph{(Ableitung der Umkehrfunktion)}\\
  Hierbei muss $f$ an den entsprechenden Punkten invertierbar sein.
  Beachten Sie die Definition des Inversen: $y = f(x) \iff f^{-1}(y) = x$, was folgt daraus für $f^{-1}(f(x))$?
\end{subex}


%%%%%%%%%%%%%%%%%%%%%%%%%%%%%%%%%%%%%%%%%%%%%%%%%%%%%%%%%%%%%%%%%%%%%%%%%%%%%%%%
\section{Bahnkurve eines Teilchens}
\label{sec:bahnkurve_eines_teilchens}

Ein Massepunkt bewegt sich auf der Bahnkurve
\[
  \mat r(t) = \colvec{v_x t + x_0 \\ y_0 \\ -\frac{g}{2} t^2 + v_z t + z_0}.
\]
Berechnen Sie seine Geschwindigkeit $\mat v(t) = \frac{\dd \mat r(t)}{\dd t}$ und Beschleunigung $\mat a(t) = \frac{\dd \mat v(t)}{\dd t}$.
Interpretieren Sie das Ergebnis.


\section{Induktionsbeweise}
Beweisen Sie die folgenden Aussagen mittels volltsändiger Induktion.
\begin{subex}
  \item $\forall n \in \NN\colon \sum_{k = n}^{2n} k = 3 \sum_{k=1}^n$
  \item Die Summe aller natürlichen Zahlen, die kleiner als $n$ sind, beträgt $\frac{n(n+1)}{2}$.
  \item $\forall n \in \NN, n \ge 4\colon 2^n \ge n^2$
\end{subex}

%%%%%%%%%%%%%%%%%%%%%%%%%%%%%%%%%%%%%%%%%%%%%%%%%%%%%%%%%%%%%%%%%%%%%%%%%%%%%%%%
\section{Die Ableitung als lineare Approximation \points{4}}
\label{sec:ableitung_als_lineare_approximation}

In der Vorlesung wurde kurz auf die Rolle der Ableitung als lokale lineare Approximation einer (im Allgemeinen nichtlinearen) Funktion $f$ eingegangen:
\[
  f(x_0 + h) = f(x_0) + f'(x_0) h + \smallO{h}.
\]
Hier steht $\smallO{h}$ für Terme höherer Ordnung als $h$, d.h.\ für $g(h) = \smallO{h}$ gilt $\lim_{h \to 0} \frac{g(h)}{h} = 0$.
Zum Beispiel ist $h^2 = \smallO{h}$, aber $\frac{1}{2}h \neq \smallO{h}$.

Betrachten Sie die Fläche eines Quadrates $A$ als Funktion der Seitenlänge $a$.
Bestimmen Sie die Ableitung $A'(a)$ mithilfe der obigen Charakterisierung, d.h.\ berechnen Sie $A(a + h)$ und identifizieren Sie die einzelnen Terme $A(a)$, $A'(a)$ und $\smallO{h}$.

Wie können sie $A'(a)$ und den Term $\smallO{h}$ geometrisch deuten (siehe auch die untere Skizze)?

\begin{center}
  \begin{tikzpicture}[scale=4]
    \draw (0,0) -- (1,0) node[midway,below] {$a$} -- (1,1) -- (0,1) -- (0,0);
    \draw[dashed] (0,0) -- (1.1,0) -- (1.1,1.1) -- (0,1.1) -- (0,0);
    \node[below] at (1.05,0) {$h$};
  \end{tikzpicture}
\end{center}

%%%%%%%%%%%%%%%%%%%%%%%%%%%%%%%%%%%%%%%%%%%%%%%%%%%%%%%%%%%%%%%%%%%%%%%%%%%%%%%%
\section{Bahnkurven}
\label{sec:bahnkurven}

Berechnen Sie die Geschwindigkeit $\mat v(t) = \deriv[\mat r]{t}(t)$ und Beschleunigung $\mat a(t) = \deriv[\mat v]{t}(t)$ der folgenden Bahnkurven.\\

\begin{subex*}
  \item $\mat r(t) = \colvec{v_x t + x_0 \\ y_0 \\ -\frac{g}{2} t^2 + v_z t + z_0}$
  \item $\mat r(t) = \colvec{R \cos \omega t \\ R \sin \omega t \\ v_z t}$
  \item $\mat r(t) = R \mathrm{e}^{-\eta t}$ \colvec{\cos \omega t \\ \sin \omega t \\ 0} 
\end{subex*}




%%%%%%%%%%%%%%%%%%%%%%%%%%%%%%%%%%%%%%%%%%%%%%%%%%%%%%%%%%%%%%%%%%%%%%%%%%%%%%%%
\section{Kontinuitätsgleichung}
\label{sec:kontinuit_tsgleichung}

Erhaltungsgrößen spielen in der Physik eine zentrale Rolle.
Als Beispiel betrachten wir eine Flüssigkeit beschrieben durch ihre Massendichte $\rho(\vec r, t)$ und Strömungsgeschwindigkeit $\vec v(\vec r, t)$ zur Zeit $t$ am Ort $\vec r$.
Wenn die Flüssigkeitsteilchen lokal erhalten ist -- d.h.\ es wird keine Masse erzeugt oder vernichtet -- gilt für ein beliebiges Raumgebiet $\Omega \subset \mathbb{R}^3$ der Massenerhaltungssatz (hier in integraler Form)
\[
  \label{eq:kont}
  \tag{$\ast$}
  \deriv{t} \int_\Omega \dd V \, \rho(\vec r, t) + \int_{\partial \Omega} \dd \vec{S} \cdot \left( \rho(\vec r, t) \vec v(\vec r, t) \right) = 0.
\]

\begin{subex}
  \item Erläutern Sie die Bedeutung der beiden Terme im Massenerhaltungssatz~\eqref{eq:kont}.
  \item Nutzen Sie den Gaußschen Integralsatz und das Argument, dass \eqref{eq:kont} für beliebige Raumgebiete $\Omega$ gilt, um die differentielle Form des Massenerhaltungssatzes (auch Kontinuitätsgleichung genannt) herzuleiten:
  \[
    \deriv{t} \rho(\vec r, t) + \nabla \cdot (\rho(\vec r, t) \vec{v}(\vec r, t)) = 0.
  \] 
\end{subex}
\end{document}
