\documentclass{scrartcl}
\usepackage{mm_ws15}

\newcommand{\sheetTitle}{Blatt 9, Abgabe 5.1.2016}
\begin{document}
\maketitle

\vspace{\baselineskip}
\emph{%
  Beachten Sie, dass der Abgabetermin in der vorlesungesfreien Zeit liegt.
}

%%%%%%%%%%%%%%%%%%%%%%%%%%%%%%%%%%%%%%%%%%%%%%%%%%%%%%%%%%%%%%%%%%%%%%%%%%%%%%%%
\section{Horizontaler Wurf \points{4}}
\label{sec:horizontaler_wurf}

Ein Massepunkt der Masse $m$ wird aus der Höhe $h$ über dem Boden mit Anfangsgeschwindigkeit $\dot{\vec r}(0) = \deriv[\vec r]{t}(0) = v_0 \vec e_x$ horizontal geworfen. Unter Vernachlässigung der Reibung folgt aus den Newtonschen Bewegungsgleichungen für die Bahnkurve $\vec r(t)$
\[
  m \ddot{\vec r}(t) = m \frac{\dd^2 \vec r}{\dd t^2}(t) = \vec F_g = - m g \vec e_y.
\]
\begin{subex}
  \item\points{3} Lösen Sie die Bewegungsgleichung für die angegebenen Anfangswerte. Dabei können Sie die $\vec e_z$ Richtung vernachlässigen und das Problem als Bewegung in 2D auffassen.
  \item\points{1} Berechnen Sie den Auftreffpunkt auf den Boden. 
\end{subex}


%%%%%%%%%%%%%%%%%%%%%%%%%%%%%%%%%%%%%%%%%%%%%%%%%%%%%%%%%%%%%%%%%%%%%%%%%%%%%%%%
\section{Separation der Variablen \points{10}}
\label{sec:separation_der_variablen}

Lösen sie die folgenden Anfangswertprobleme mit der Methode der Separation der Variablen.

\begin{subex}
  \item\points{3} Unbeschränktes Wachstum $\deriv[N(t)]{t} = \beta N(t)$ mit $N(0) = N_0 > 0$ und $\beta > 0$.
  \item\points{3} Beschränktes Wachstum $\deriv[N(t)]{t} = \beta N(t) - \alpha t N(t)$  mit $N(0) = N_0 > 0$ und $ \alpha, \beta > 0$.
  \item\points{3} Barometrische Höhenformel: $\deriv[p(h)]{h} = - \frac{p(h) M g}{R(T_0 - ah)}$ mit $p(h_0) = p_0$ und $M, g, R > 0$.
  \item\points{1} Skizzieren Sie die Lösungen von a) und b) und diskutieren Sie ihr Verhalten für kleine $t$ und $t \to \infty$.
\end{subex}

\begin{remark}{Hinweis}
  Separation der Variablen ist eine Lösungmethode für Differentialgleichungen 1.\ Ordnung der Form
  \[
    \deriv[x(t)]{t} = g(t) h(x(t)).
  \]
  Die in der Vorlesung besprochenen Umformungen zur Lösung der DGL lassen sich gut als \quotes{Multiplikation mit $\dd t$} und anschließende Integration über $x$ und $t$ merken:
    \[
    \deriv[x(t)]{t} = g(t) h(x(t)) \implies \frac{\dd x}{h(x)} = g(t) \, \dd t \implies \int \frac{1}{h(x)}\,  \dd x = \int g(t) \, \dd t.
  \]
  Diese Regel soll aber nur als Gedankenstütze dienen, da Multiplikation mit $\dd t$ hier nicht mathematisch definiert ist.
\end{remark}


%%%%%%%%%%%%%%%%%%%%%%%%%%%%%%%%%%%%%%%%%%%%%%%%%%%%%%%%%%%%%%%%%%%%%%%%%%%%%%%%
\section{Variation der Konstanten \points{6}}
\label{sec:variation_der_konstanten}

Lösen Sie die folgenden inhomogenen Differentialgleichungen.

\begin{subex}
  \item\points{3} $\deriv[x(t)]{t} + \alpha x(t) = \beta t$ für $t \ge 0$ mit $x(0) = 1$ und $\alpha, \beta > 0$.
  \item\points{3} $t \deriv[y(t)]{t} + 2 y(t) = t^2 - t + 1$ für $t \ge 1$ mit $y(1) = 1$.
\end{subex}

\end{document}
