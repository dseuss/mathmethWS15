\documentclass{scrartcl}
\usepackage{mm_ws15}


\newcommand{\sheetTitle}{Blatt 6, Abgabe 1.12.2015 12:00}
\begin{document}
\maketitle

\vspace{\baselineskip}
\emph{%
  Gegebenfalls können aus Zeitgründen nicht alle Aufgaben in der Übung vorgerechnet werden.
  Von jedem Aufgabentyp sollte aber mindestens eine Teilaufgabe behandelt werden.
  Die Ergebnisse (ohne Rechenwege) werden nach den Übungen auf die Übungsseite gestellt.
}


%%%%%%%%%%%%%%%%%%%%%%%%%%%%%%%%%%%%%%%%%%%%%%%%%%%%%%%%%%%%%%%%%%%%%%%%%%%%%%%%
\section{Taylorreihen \points{6}}
\label{sec:taylorreihen}

Bestimmen Sie die Taylorreihen $f(x) = \sum_{k=0}^\infty a_k x^k$ der folgenden Funktionen indem Sie die Koeffizienten $a_n$ explizit angeben.\\

\begin{subex*}
  \mitem{f(x) = a^x} 
  \mitem{f(x) = \frac{1}{(1 + x)^2}} 
  \mitem{f(x) = \ee^{-x^2}}
\end{subex*}

%%%%%%%%%%%%%%%%%%%%%%%%%%%%%%%%%%%%%%%%%%%%%%%%%%%%%%%%%%%%%%%%%%%%%%%%%%%%%%%%
\section{Partielle Integration \points{8}}
\label{sec:partielle_integration}

Berechnen Sie mit Hilfe partieller Integration.
\begin{flalign*}
  \mathbf{a)} &\quad  \int_0^1 \dd x\, x \, \ee^{-2 x} &
  \mathbf{b)} &\quad \int_{0}^{\pi/2} \dd x\, \sin(x)  \cos(x) & \\
  %\mathbf{c)} &\quad \int_0^1 \dd x\, (1 + x) \, \ee^{ x}  &\\
  \mathbf{c)} &\quad \int_1^2  \dd x\,\frac{\log(x)}{x^2}  &
  \mathbf{d)} &\quad \int  \dd x\,\ee^{a x} \cos (b x) & 
\end{flalign*}


%%%%%%%%%%%%%%%%%%%%%%%%%%%%%%%%%%%%%%%%%%%%%%%%%%%%%%%%%%%%%%%%%%%%%%%%%%%%%%%%
\section{Integration durch Substitution \points{8}}
\label{sec:substitution}

Berechnen Sie mittels Substitution
\begin{flalign*}
  \mathbf{a)}&\quad \int_0^1 \dd x\,  x \, \ee^{x^2} &
  \mathbf{b)}&\quad \int_0^2 \dd x\,  \frac{1}{\sqrt{7 - 3x}}   & \\
  \mathbf{c)}&\quad \int  \dd x\, x^2  \sqrt{2 x^3 + 4}   & 
  \mathbf{d)}&\quad \int_{a}^{b} \dd x\,  x  \cos(x^{2}+1)   & 
  %\mathbf{c)}&\quad \int_0^1 \dd x\,  \frac{x}{x^2+1}  \\
\end{flalign*}


%%%%%%%%%%%%%%%%%%%%%%%%%%%%%%%%%%%%%%%%%%%%%%%%%%%%%%%%%%%%%%%%%%%%%%%%%%%%%%%%
\section{Mehrfachintegrale \rom{1} \points{6}}
\label{sec:mehrfachintegrale1}

Berechnen Sie die folgenden Mehrfachintegrale.
\begin{flalign*}
  \mathbf{a)} \, &\quad  \int_0^a \dd x \int_0^b \dd y \int_0^h \dd z\, \ee^{-\alpha z}&  
  \mathbf{b)} \, &\quad \int_{0}^{\pi/2} \dd x \int_{0}^{x} \dd y\,\, \sin(x) \sin(y)& \\
  \mathbf{c)} \, & \quad \int_1^2 \dd y \int_1^{y^2} \dd x \log(x+y) & &
\end{flalign*}


%%%%%%%%%%%%%%%%%%%%%%%%%%%%%%%%%%%%%%%%%%%%%%%%%%%%%%%%%%%%%%%%%%%%%%%%%%%%%%%%
\section{Mehrfachintegrale \rom{2}  \points{2}}
\label{sec:mehrfachintegrale2}

Betrachten wir nun als Anwendung von Mehrfachintegralen einen Quader, dessen Breite $a$ (in $x$-Richtung), Länge $b$ (in $y-$Richtung) und Höhe $c$ (in $z-$Richtung) beträgt. Die Dichte (also Masse pro Volumen) dieses Quaders ist nicht homogen, sondern hängt von $x,y$ und $z$ ab - stellen Sie sich zum Beispiel ein Gemisch aus Sand und Kieselsteinen im Inneren des Quaders vor. Hier soll sie über die Funktion
\begin{align*}
  \rho(x,y,z)=\rho_0\frac{(2x+3y)z^2}{V}
\end{align*}
beschrieben werden. Dabei ist $\rho_0$ eine Konstante der Einheit Masse pro Volumen und $V$ das Volumen des Quaders. Berechnen Sie die Masse $M$ des Quaders. Welches Ergebnis erhielten Sie, falls die Dichte konstant wäre, also $\rho(x,y,z)=\rho_0$ gelten würde?


\end{document}