\documentclass{scrartcl}
\usepackage{mm_ws15}

\newcommand{\sheetTitle}{Blatt 14, Abgabe 9.2.2015}
\begin{document}
\maketitle
\vspace{1em}
\emph{%
  Alle Aufgaben auf diesem Blatt sind freiwillig abzugeben -- die erreichten Punkte zählen als Zusatzpunkte.
  Nutzen Sie die letzte Übung außerdem dazu, gezielt Fragen in Vorbereitung auf die Klausur zu stellen.\\
  Die Übungen zum 13.\ Blatt am Donnerstag, den 4.2.\ fallen wegen Weiberfastnacht aus.
  Stattdessen werden in der Fragestunde am 5.2.\ die Aufgaben gemeinsam besprochen.
}
%%%%%%%%%%%%%%%%%%%%%%%%%%%%%%%%%%%%%%%%%%%%%%%%%%%%%%%%%%%%%%%%%%%%%%%%%%%%%%%%
\section{Vektorielles Oberflächenintegral I\,(8 ZP)}
\label{sec:gau_scher_integralsatz}

Eine Punktladung mit elektrischer Ladung $Q$ erzeugt im Vakuum das elektrische Feld
\[
  \vec E(\vec r) = \frac{Q}{4\pi \epsilon_0} \cdot \frac{\vec r}{r^3}
\]
mit $r = \Vert \vec r \Vert$.
Berechnen Sie den elektrischen Fluss $F_a$ von $\vec E$ durch die Oberfläche einer Kugel $\Omega$ mit Radius $a$, die im Ursprung zentriert ist
\[
  F_a = \int_{\partial\Omega} \dd \vec S \cdot \vec E(\vec r, t).
\]


%%%%%%%%%%%%%%%%%%%%%%%%%%%%%%%%%%%%%%%%%%%%%%%%%%%%%%%%%%%%%%%%%%%%%%%%%%%%%%%%
\section{Vektorielles Oberflächenintegral II\,(8 ZP)}
\label{sec:vektorielles_oberfl_chenintegral_ii}

Berechnen Sie den Fluss des Vektorfeldes
\[
  \mat F(\mat r) = \colvec{z^2 \\ 2zy \\ x^2}
\]
durch den Zylindermantel $Z = \{ (x,y,z) \in \RR^3 \colon x^2 + y^2 = 4, 0 \le z \le 2 \}$.


%%%%%%%%%%%%%%%%%%%%%%%%%%%%%%%%%%%%%%%%%%%%%%%%%%%%%%%%%%%%%%%%%%%%%%%%%%%%%%%
\section{Differentialoperatoren\,(4 ZP)}
\label{sec:differentialoperatoren}

 Berechnen Sie die Divergenz und Rotation der Vektorfelder aus den vorhergehenden Aufgaben.

\sepline[.75\textwidth]
%%%%%%%%%%%%%%%%%%%%%%%%%%%%%%%%%%%%%%%%%%%%%%%%%%%%%%%%%%%%%%%%%%%%%%%%%%%%%%%%
\section*{Hinweise zur Klausur}
\label{sec:hinweise_zur_klausur}

Die Klausur findet am Donnerstag, den 25.2.2016 in der Zeit 9:00 -- 12:00 in den Hörsälen I+II statt.
Die Aufteilung auf die Räume finden Sie kurz vorher auf der Übungsseite.
Bitte melden Sie sich rechtzeitig über KLIPS für die Klausur an.
Falls es Probleme bei der Anmeldung geben sollte, wenden Sie sich bitte an das Prüfungsamt -- erscheinen Sie aber auf jeden Fall bei der Klausur.

Die Klausur wird -- ähnlich wie die Testklausur -- zum großen Teil aus ``Rechenaufgaben'' bestehen.
Als Vorbereitung empfehle ich Ihnen Aufgaben, die den Rechenaufgaben in den Übungen ähneln, \emph{selbständig} zu bearbeiten.
Für diesen Zweck gibt es Aufgabensammlungen (z.B.\ in der Bibliothek).
Zu Beginn der Klausur lesen Sie alle gestellten Aufgaben gründlich durch und beginnen mit der einfachsten.


\end{document}
